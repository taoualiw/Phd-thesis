\DontNumberThisInToc
\DontFrameThisInToc
\ChapterNoNumberCitation{Conclusion g{\'e}n{\'e}rale et perspectives}{La stratégie est de construire un système cognitif, non à partir de symboles et de règles, mais à partir de constituants simples qui peuvent dynamiquement être reliés les uns aux autres de manière très dense. Un tel système ne requiert dont pas d'unité centrale de traitement pour contrôler son fonctionnement.(...) Ce transfert de règles locales à la cohérence globale est le cœur de ce qu'il était convenu d'appeler auto-organisation au temps de la cybernétique. Aujourd'hui plutôt que d'auto-organisation on préfère parler de propriétés émergentes globales.}{Francisco J. Varela (1996). Invitation aux sciences cognitives.}{10cm}




Dans ce travail de thèse nous avons exploré les avancées que l'étude des systèmes neuronaux peut apporter à la compréhension de la thématique ``perception versus action''. En effet, Les réseaux de neurones artificiels et les modèles de champs neuronaux doivent leur intérêt à la capacité qu'ils ont à modéliser une variété importante de structures neuronales (colliculus, cervelet, noyaux sous$-$corticaux, cortex sensoriel, associatif ou moteur) et à expliquer certains phénomènes du vivant, en allant de l'échelle neuronale à l'échelle comportementale. Les constituants essentiels de ces modèles (à savoir les flux d'informations entrants et sortants, la connectivité interne et l'organisation temporelle) peuvent s'adapter à la diversité des architectures de ces structures et offrent alors un outil expérimental efficace pour les simuler et mieux les comprendre. \\%intro

Nous nous sommes intéressés à identifier les caractéristiques du calcul neuronal à exploiter pour permettre, à long terme, le développement de systèmes ``enactifs'' dans la perspective des affordances de \cite{Varela:1993} et à court terme de comprendre le lien entre structures et fonctions dans le cerveau. Nous avons cherché, en particulier, à comprendre les propriétés émergentes de l'interaction entre des unités élémentaires en combinant deux approches. Une première approche consiste à s'inspirer des données biologiques et des expériences comportementales pour définir les paramètres de base et poser des hypothèses sur les rôles et fonctions des différentes structures et différents flux d'information. La deuxième approche consiste, inversement, à partir des modèles pour essayer d'expliquer ce qui se passe dans le monde réel, en donnant des prédictions et des schémas fonctionnels qui inspirent et dirigent les biologistes dans leurs investigations.\\%intro

Nous avons essayé de répondre essentiellement aux problématiques suivantes:
\begin{itemize}
\item D'un point de vue calculatoire: 

\begin{itemize}
\item[$\bullet$]Quelles sont les hypothèses à considérer à propos de l'organisation spatiale et temporelle des réseaux de neurones ?
\end{itemize}
\item D'un point de vue fonctionnel: 
\begin{itemize}
\item[$\bullet$]Comment obtenir, à partir de règles locales simples dans une sous-population homogène et d'un traitement séquentiel, des propriétés et des comportements complexes (en particulier une sélection de l'action)?
\item[$\bullet$]Comment une prise de décision est assurée, à large échelle, lorsque des boucles différentes et des flux hétérogènes viennent biaiser un tel traitement? \\
\end{itemize}
\end{itemize}


Nous avons d'abord examiné le problème calculatoire en abordant les caractéristiques principales du cadre de modélisation. En effet, de nombreux modèles de réseaux de neurones sont régis par des équations différentielles qui sont en général très complexes surtout lorsqu'on cherche à faire des modèles biophysiques bien précis. Il est donc nécessaire de s'intéresser à l'étude de simplifications qui permettent tout de même de capturer les propriétés essentielles du calcul neuronal, dans le cadre des fonctions que l'on cherche à réaliser. Deux problèmes s'imposent. D'abord, la résolution numérique nécessite le passage de la description continue à une description discrète (dans l'espace et le temps). Ensuite, il est implicitement supposé que le calcul est \textit{synchrone} entre les différentes unités, et c'est le cas dans la plupart des réseaux de neurones artificiels d'aujourd'hui. Cette position est contestable d'un point de vue biologiste, d'o\`u l'intérêt d'examiner le mode d'évaluation asynchrone dans le cadre des neurosciences computationnelles.\\%synthèse

Dans ce contexte, nous avons examiné les problèmes de discrétisation liés à la résolution numérique, dans le cadre particulier de la théorie des champs neuronaux continus (CNFT) \cite{Wilson:1973, Amari:1977} dans une approche connexionniste (le comportement du réseau émerge d'une règle locale). Nous avons montré que lors de la simulation d'un système à temps continu régie par une équation différentielle, le choix du schéma de discrétisation est primordial pour la détermination de la précision des trajectoires de l'évolution même si l'état final est souvent conservé. En plus, plus le schéma de discrétisation s'approche du schéma continu, plus la mise en oeuvre est compliquée et coûteuse en termes de ressources de calcul. Il en résulte que le niveau de détail et de précision est déterminé par l'objectif des travaux de simulation qui impose le rapport qualité/coût. Ensuite, nous avons étudié dans quelle mesure nous pouvons éliminer cette horloge centrale et mettre en oeuvre un calcul plutôt \textit{asynchrone} dans le cadre des champs neuronaux dynamiques. Nous nous sommes basés sur des résultats connus des systèmes discrets asynchrones en les reliant au domaine des neurosciences computationnelles. Deux hypothèses principales se sont avérées importantes: D'une part, les retards des mises à jour entre les différentes unités devraient être bornés par un nombre fini constant et uniforme, d'autre part, il y a une limite maximale uniforme, entre deux mises à jour pour une unité donnée. Chaque unité doit fournir une mise à jour au moins une fois pendant un intervalle de temps de longueur fini pour garantir qu'il n'y a pas d'unités qui ne se mettent jamais à jour. \\%résultats

Un tel schéma de calcul discret et asynchrone est en faveur d'une implémentation informatique distribuée, fortement souhaitée dans notre cadre. En plus de ces deux aspects, la CNFT présente un exemple pertinent d'organisation spatiale des unités (cartes continues) et des profils de connexions (différence de gaussiennes). Plusieurs résultats empiriques montrent que la CNFT offre un paradigme de calcul qui permet, à travers une compétition entre les différentes unités du réseau, de faire émerger une décision collective robuste, précise et rapide. En plus, cette décision est adaptative puisque ce formalisme permet de mettre en oeuvre des schémas d'apprentissage et d'auto-organisation. Il en résulte donc les quatre hypothèses suivantes qui permettent d'avoir un calcul neuronal permettant de modéliser au mieux les comportements complexes: calcul local, adaptatif, discret et asynchrone. Ces caractéristiques peuvent être mises en oeuvre dans le cadre de la théorie de la CNFT. \\%discussion

Mais comme notre étude le montre, la mise en oeuvre d'un calcul asynchrone et discret rend à la fois la modélisation et l'implémentation plus compliquées. C'est pourquoi nous devrions alors prendre des précautions supplémentaires lors de la description d'un système en utilisant les équations continues. En particulier, au niveau de la modélisation mésoscopique, il peut être intéressant, comme perspective de travail, d'utiliser un formalisme basé sur les événements, comme il s'agit d'un paradigme bien défini qui prend en considération le fait que non seulement le traitement, mais aussi le calendrier sont entièrement distribués. \\%perspectives


Pour traiter la deuxième problématique, et en se basant sur le formalisme offert par la CNFT, nous avons proposé un modèle minimaliste de colliculus supérieur sur la base d'un large ensemble de données biologiques. Ce modèle a été alors conçu en s'appuyant sur un minimum d'hypothèses dans un cadre de calcul numérique distribué. La population homogène considérée est la carte colliculaire et le flux séquentiel correspond au flux visuel arrivant de la rétine (aucun feedback et aucune entrée supplémentaire à l'entrée visuelle ne sont considérés). Cette carte est supposée avoir pour rôle l'encodage du but, c'est à dire la direction de l'action résultante de la perception (la saccade). \\ % synthèse

Le comportement du modèle, en particulier la sélection, à travers les différentes expériences, est une propriété émergente d'une règle de calcul locale et simple. L'unicité et la stabilité de la sélection sont assurées par les connexions latérales au sein de la carte. Autrement dit, le profil de connectivité latérale proposé par la CNFT assure la fonction du ``winner-takes-all'' et la stabilité de la solution finale. En outre, si on examine de plus près le processus de sélection qui se réalise lorsque le modèle reçoit en entrée deux stimuli identiques (mais à deux endroits différents), le choix des stimuli les plus proches de la région fovéale s'explique par la magnification corticale donc par la topographie des projections qui arrivent sur la carte. En effet, la saillance d'un stimulus est déterminée par l'activation colliculaire initiale qu'il évoque. C'est pourquoi un stimulus qui se projette sur la région fovéale de la rétine et par la suite la région rostrale du colliculus est considéré comme plus saillant. Nous avons examiné d'autres comportements du modèle comme l'altération de l'exactitude en cas d'inhibition locale imposée et les simulations ont montré des résultats en accord avec les résultats biologiques.\\%résultats 

Ce modèle nous a permis d'expliquer plusieurs comportements en partant d'un minimum d'hypothèses et de règles locales. En particulier, la sélection qu'on peut qualifier d'intrinsèque (par défaut) est donc très liée à la disposition spatiale et physique des unités et des connexions. L'``intelligence'' du système est donc ancrée dans son instanciation physique (même si elle est simulée dans notre cas) et le traitement séquentiel est suffisant pour engendrer un choix d'une action (saccade vers un emplacement donné) à partir d'une perception visuelle. C'est donc la répartition spatiale des projections inhibitrices et excitatrices, latérales et verticales entre les cartes régies par des équations locales simples qui peuvent être à l'origine de certains comportements complexes ou imprévisibles vu qu'il s'agit d'un codage par population. \\%discussion

Avec ce modèle nous avons montré comment à partir d'un stimulus visuel unique (une perception) une saccade adéquate (une action) est encodée automatiquement. Nous avons aussi montré que la sélection d'une cible, quand plusieurs stimuli sont présentés, est déterminée par l'activation initiale la plus forte influencée en particulier par la taille, la position et l'intensité des stimuli. Mais en général, une saccade n'est pas forcément guidée par la saillance visuelle du stimulus. Comme reporté dans le deuxième chapitre, les saccades peuvent être motivées comme dans les tâches d'anti-saccades et de saccades vers des cibles mémorisées. Notre modèle est incapable, en son état actuel, de rendre compte de tels comportements. D'o\`u l'intérêt de s'intéresser aux mécanismes de motivations et de sélection dépendant du contexte, impliquant intuitivement plusieurs boucles et flux d'informations puisque ces mécanismes sont basés sur des interactions actives avec l'environnement. \\%TRANSITION

Nous avons donc considéré le cas plus général qui prend en compte plusieurs boucles et flux d'informations, ce qui correspond à la troisième problématique. Nous nous sommes intéressés au comportement de sélection guidée par le contexte. En particulier, nous nous sommes demandés comment il est possible de diriger l'attention vers des points non-saillants pour permettre l'exploration de l'image. Penser aux mécanismes de motivation pour guider les saccades et étudier la sélection active nous a amené à nous intéresser aux ganglions de la base, structure supposée être impliquée dans les mécanismes de sélection de l'action et de séquences motivées.\\%synthèse

Notre avons proposé un modèle des ganglions de la base pour la sélection de l'action, inspiré des modèles \gls{gpr} et \gls{cbg} et qui conserve le mécanisme basique de sélection intrinsèque par désinhibition du plus fort. En effet, les différents flux amenés par les boucles récurrentes permettent de mettre en compétition une excitation diffuse et une inhibition locale à travers les différentes voies qui transmettent l'information entre les structures d'entrée et les structures de sortie. Une différence principale par rapport aux modèles proposés par \cite{Gurney:2001a, Girard:2008} est que nous proposons un mécanisme d'évaluation de la saillance qui donne plus de liberté dans la détermination du choix de l'action, en considérant en particulier le retour thalamique vers le striatum et l'apprentissage du contexte. C'est ainsi qu'une action moins saillante peut être choisie. Si on fait le parallèle avec l'exploration saccadique, cela correspond à diriger l'attention vers un point qui n'est pas mis en valeur et peut expliquer comment il est possible de faire des anti-saccades par exemple. L'apprentissage du contexte a été implémenté par un mécanisme adaptatif qui met en exergue le rôle possible des ganglions de la base dans l'encodage du contexte et permet donc de moduler la sélection intrinsèque prédéfinie par la topographie, l'architecture et les habitudes.\\%résultats

A travers ce modèle nous avons montré que la compétition peut être assurée par l'interaction entre différentes boucles récurrentes (au sein de la structure elle-même ou avec les autres structures) et que l'intégration de plusieurs flux d'informations hétérogènes peut mettre en oeuvre un mécanisme adaptatif d'apprentissage. De plus, le fait d'intégrer un feedback thalamique qui permet de garder une trace de la sortie au niveau de l'entrée assure à une échelle mésoscopique la fonction d'une mémoire de travail. Cette mémoire de travail vient altérer la perception et donc l'influencer. Ceci nous rapproche du cadre général de l'énaction annoncé dans l'introduction. En outre, nous avons introduit un mécanisme d'apprentissage particulier, qui ne correspond pas à un apprentissage d'associations contexte-réponse. Il s'agit plutôt d'un mécanisme de création de contexte, donc de création dynamique de représentations internes en accord avec l'hypothèse qui s'oppose à la métaphore informatique du cerveau.\\%discussion

En résumé, nous avons examiné, à cours de cette thèse, les paradigmes de calcul permettant de mettre en oeuvre des comportements intéressants. Cette étude nous a permis de comprendre, d'une part, les liens entre les structures dans le cerveau et leurs fonctions. D'autre part, elle nous a fourni des éléments de réponses aux problématiques concernant la maîtrise des propriétés du calcul neuronal à mettre en oeuvre dans la modélisation des systèmes sensori-moteurs. Mais nous n'avons pas pu aborder tous les travaux nécessaires à l'étude complète de la modélisation d'un système énactif (visuo-moteur). Plusieurs points de discussion restent à étudier et plusieurs problèmes restent à résoudre, afin de compléter les modèles que nous proposons. \\%synthèse

En particulier, une fois la cible choisie, une saccade est initiée par l'activation des générateurs moteurs sur lesquels le colliculus supérieur se projette. Il serait intéressant d'ajouter la partie d'exécution motrice en aval du modèle du colliculus que nous proposons, pour avoir le passage complet de la perception à l'action. En effet, les mouvements saccadiques résultent d'un mécanisme de génération d'impulsions. De nombreux modèles de génération de saccades supposent que les trajectoires des saccades sont stéréotypés \cite{VanGisbergen:1985, Tweed:1985, Grossberg:1986, Scudder:1988, Becker:1990, Moschovakis:1994, Nichols:1995, Quaia:1997, Breznen:1997, Gancarz:1998}. En particulier, la vitesse et l'amplitude d'une saccade sont déterminées respectivement par l'intensité et l'intégrale dans le temps de l'activation. Mais d'autres données expérimentales montrent que les saccades ont des trajectoires assez variables \cite{Erkelens:1995}. Donc il serait pertinent d'examiner des modèles du générateur moteur de saccades pour compléter le nôtre.  \\%perspectives

En outre, une autre problématique s'impose si on considère le cas général d'une séquence de saccades dans le but d'une exploration motivée d'une scène visuelle. En effet, une fois qu'une saccade est réalisée vers un stimulus saillant dans l'image visuelle perçue, selon notre modèle la projection de ce stimulus sur la rétine est placée sur la fovéa, ce qui active donc la zone rostrale du colliculus. Pour effectuer une deuxième saccade, un problème de \textit{remapping} s'impose pour recentrer la projection du champ visuel dans la rétine sur le stimulus choisi. En plus, il est intuitivement nécessaire d'inhiber l'activation colliculaire rostrale due à la fixation. Cette activation l'emporte sur les activations qui peuvent être causées par des stimuli hors de la zone fovéale comme montré dans les simulations de la sélection de stimuli à différentes excentricités. D'o\`u le besoin d'un retour (feedback) moteur qui, une fois implémenté et sans ajouter des entrées extérieures, permettra de réaliser une séquence de saccades sur le seul critère de la proximité de la zone de fixation, si on considère des stimuli identiques dans l'image perçue. D'o\`u l'intérêt de penser à comment intégrer un mécanisme d'exploration motivée. Il serait intéressant d'explorer et adapter la stratégie infotaxis \cite{Vergassola:2007}, qui a été initialement conçue pour la navigation olfactive, dans le domaine visuel. Il s'agit de se déplacer (vers une source) en maximisant l'information olfactive collectée dans l'environnement via des comportements d'exploitation et d'exploration. Dans le cadre visuel, l'équivalent est de diriger les saccades vers les endroits de la scène visuelle o\`u il y a le maximum d'information. \\ %perspectives

En plus des problèmes cités précédemment qui restent à résoudre pour compléter le modèle du colliculus, une perspective de ce travail serait de considérer la boucle sous-corticale reliant le colliculus, les ganglions de la base et le thalamus. En effet, les \gls{bg} agissent de deux manières sur les mouvements oculaires. D'abord, ils facilitent l'initiation des saccades volontaires générées dans un contexte de comportements appris, de prédiction ou de récompense retardée. Cette facilitation est réalisée via le cortex frontal par la désinhibition des couches intermédiaires du \gls{sc} qui sont sous inhibition tonique (empêchant le déclenchement de saccades). Le colliculus supérieur est alors ``libéré'' et la génération de saccades est rendue possible. Ensuite, par leur retour direct vers \gls{sc}, ils permettent la génération directe de saccades réflexes et peuvent empêcher les non souhaitées \cite{Leigh:2006}. Il s'avère donc pertinent de considérer la boucle basalo-colliculaire comme suite du modèle.\\  %perspectives

Il serait de même pertinent de considérer l'hypothèse de \cite{Haber:2003} qui s'oppose au postulat de la séparation des circuits moteur, cognitif et limbique incluant les ganglions de la base. Les auteurs proposent une interaction séquentielle entre les différents circuits. D'abord, les émotions et la motivation (le système limbique) initient les comportements. Ensuite, les fonctions exécutives (le système cognitif) organisent et planifient la stratégie de l'exécution de l'action (le système moteur). Chaque composante de l'information serait donc renvoyée vers son circuit d'origine (la sortie est renvoyée vers la partie corticale stimulée dans le circuit concerné) mais également transférée au circuit suivant.\\ %perspectives

En effet, dans notre modèle nous avons montré que les ganglions de la base peuvent participer à la création d'une représentation du contexte; cette représentation peut correspondre à l'entrée du circuit cognitif. Ceci permettrait peut-être de retrouver la structure en spirale proposée par \cite{Haber:2003}. Les processus de prise de décision motrice seraient ainsi influencés par les signaux limbiques puis cognitifs, permettant à l'organisme de s'adapter de manière appropriée à son environnement. \\  %perspectives

Enfin pour conclure, cette démarche rejoint la théorie de l'énaction qui se base sur le fait que notre représentation de l'environnement est dynamique, que la perception a pour but d'aider à agir et que l'action à son tour structure la perception. Dans la mesure o\`u les modèles servent en particulier à valider les idées (et en prédire d'autres), notre travail de thèse nous a permis de comprendre que la perception n'a pas forcément pour vocation de reconstruire une représentation du monde extérieur. En particulier, la perception visuelle ne sert pas à reproduire une représentation interne de l'image visuelle perçue. Il en résulte qu'un modèle de transformation sensori-motrice ne peut pas être complet si on ne considère pas le retour moteur sur le sensoriel et si on néglige l'influence des entrées endogènes (motivation) et exogènes (contexte).\\%conculusion

On parle donc d'un couplage opérationnel entre l'organisme et son environnement, entre les processus sensoriels et moteurs. La perception et l'action deviennent donc indissociables \cite{Varela:1993}. Selon cette position, la cognition ne se réduit donc pas à un traitement automatique séquentiel de l'information (stimulus-représentation interne-action) qui suppose que nous avons un enregistrement de données issues d'un monde prédéfini mais à une complémentarité entre le flux perceptif (l'expérience perceptive) et l'historique des actions (l'expérience active) qu'accomplit un être dans le monde.\\%conclusion

