\DontNumberThisInToc
\DontFrameThisInToc
\ChapterNoNumberCitation{Introduction g{\'e}n{\'e}rale}{Poets say science takes away from the beauty of the stars - mere globs of gas atoms. I too can see the stars on a desert night, and feel them. But do I see less or more? The vastness of the heavens stretches my imagination… It does not do harm to the mystery to know a little about it. For far more marvelous is the truth than any artists of the past imagined it.}{Richard Feynman}{10cm}

La compréhension du fonctionnement du corps humain et singulièrement du cerveau a depuis toujours passionné le monde de la recherche. Le cerveau chez les vertébrés reçoit des afférences internes de tout l'organisme et externes par l'interaction du corps avec son environnement. La principale fonction du cerveau est de contrôler les actions de l'organisme en interprétant des entrées sensorielles qui lui parviennent. Ces entrées sensorielles peuvent moduler un processus en cours, déclencher une réponse (action) immédiate ou rester mémorisées pour un besoin futur. Ce traitement des informations est assuré par des mécanismes de réception, d'intégration et de renvoi de signaux électriques à l'échelle microscopique, se traduisant par des flux entre différentes régions et structures corticales et sous-corticales à l'échelle macroscopique. En effet, le cerveau est organisé en sous-systèmes fonctionnels spécialisés dans le traitement de certains aspects particuliers de l'information. Cette organisation dote le cerveau d'un ensemble de capacités (fonctions) cognitives (la mémoire, l'attention, le langage ...) assurant le déroulement de  processus mentaux entre une stimulation et une réponse, autrement dit entre une perception et une action.\\

La théorie dominante (la métaphore de l'ordinateur) de l'intelligence artificielle postule que notre cerveau agit de manière séquentielle et peut donc être comparé à une machine automatique (entrées, traitement d'informations, représentation, sortie). Selon cette théorie, le cerveau produirait des représentations internes du monde et donnerait lieu à l'expérience de voir quand certaines représentations mentales sont activées, sans dépendre dans ses décisions ni du contexte extérieur ni de l'expérience du corps qui le contient. Dans le système visuel par exemple, le rôle de l'œil est assimilé à celui d'une caméra. Il s'agit alors de transmettre au cerveau  une image mentale de l'environnement en deux dimensions. Le cerveau effectuerait ensuite des traitements sur cette image et élaborerait une représentation, en la comparant avec des représentations mémorisées par exemple, pour renvoyer une réponse. Un exemple qui échappe à cette hypothèse, la perception de la profondeur, a été étudié par le psychologue  James J. Gibson qui propose que la perception n'est pas un acte figé dans le temps ou dans l'espace. Il s'agit plutôt d'un flux d'informations modifiable par les actions. Il parle donc de perception active (\textit{``perceiving is an act of attention, not a triggered impression.''}) \cite{Gibson:1986}.\\

Cette position se rapproche de la théorie de l' ``énaction'' introduite par \textit{Francisco Varela} dans \cite{Maturana:1987} et défendue ensuite par \textit{Alva Noë} dans son livre \textit{Action in Perception}, paru en 2004  (\textit{`` nous énactivons notre perception ''}) \cite{Noe:2004}. Selon lui, l'expérience perceptive ne se réalise qu'à partir du moment o\`u notre corps agit sur le monde physique par ses mouvements et son savoir-faire en fonction du contexte. La perception (consciente) est donc qualifiée de ``tactile'': comme un aveugle reconnaît un objet en le manipulant par les mains (le toucher), on peut imaginer qu'une personne voyante reconnaît un objet en le manipulant d'une manière similaire, mais par les yeux via les saccades oculaires. C'est alors la succession et la nature des actions effectuées et les résultats de l'expérience motrice qui donneront la conception de l'objet. Il s'agit donc d'une conception de la ``cognition incarnée'' qui implique le cerveau, le corps et le contexte environnemental \cite{Noe:2004,Regan:2001}.\\

Pour étudier cette approche, le formalisme de modélisation à base de réseaux de neurones présente plusieurs intérêts puisqu'il permet de prendre en compte des flux d'entrées asynchrones et hétérogènes (traitement non séquentiel), un codage par populations (calcul distribué) et des boucles en interaction permettant de créer des représentations dynamiques (images non figées). En effet, il y a beaucoup plus de connexions neuronales internes entre les différentes structures du cerveau que de connexions avec les organes sensoriels (avec l'environnement extérieur). L'activité du cerveau n'aurait donc pas pour but de reconstituer un monde extérieur avec des propriétés prédéfinies mais plutôt de l'interpréter au sein d'un réseau neuronal \cite{Varela:1993} qui aurait pour fonction de construire des représentations dynamiques en fonction des différents flux sensoriels et moteurs et de leurs interactions.\\ 

En s'inscrivant dans cette conception de la cognition et pour bien maîtriser le calcul neuronal permettant de comprendre les relations entre les structures dans le cerveau et leurs fonctions, plusieurs problèmes restent à résoudre au niveau calculatoire et au niveau fonctionnel. Au niveau calculatoire, ces flux d'information doivent pouvoir se déployer de manière indépendante, sans suivre le cadencement imposé par le cadre stimulus-réponse. Au niveau fonctionnel le plus élémentaire, des populations de neurones doivent pouvoir faire émerger des représentations suffisamment riches pour permettre la transformation de représentations sensorielles spatiales en actes moteurs temporels mais suffisamment simples pour rester exploitables par d'autres processus mentaux de niveaux supérieurs. Au niveau fonctionnel le plus global, des flux sensoriels, moteurs et motivationnels doivent pouvoir interagir en vue d'une décision globale reposant sur des évaluations locales. Dans le cadre de cette thèse, nous avons voulu nous intéresser à ces trois problématiques principales et essentielles à l'approche que nous défendons, en nous restreignant toutefois au cadre visuo-moteur:\\

\begin{itemize}

\item[$\bullet$] D'abord, nous nous sommes concentrés sur les problèmes liés aux paradigmes de calcul et formalismes de modélisation à utiliser pour le développement de systèmes énactifs. En effet, les réseaux de neurones artificiels et les modèles de champs neuronaux  reposent sur des bases mathématiques solides issues du domaine des systèmes dynamiques et sur un ensemble d'hypothèses bien établies pour ce qui concerne la plausibilité biologique de leurs mécanismes. Dans ces deux domaines scientifiques, des progrès importants ont été permis par l'apport de l'informatique, aussi bien à travers les simulations à large échelle que par l'étude du paradigme de calcul distribué sous$-$tendu par ces modèles. Nous avons en particulier examiné des hypothèses sur l'organisation spatiale et temporelle de ces réseaux, dans un contexte où l'utilisation d'une horloge globale est proscrite mais sera mise en question.\\

\item[$\bullet$] Ensuite, une question primordiale s'est imposée: comment obtenir à partir de règles locales simples des propriétés et des comportements complexes (en particulier des décisions)?  En effet, le neurone est la cellule nerveuse qui représente la brique élémentaire du traitement de l'information et du développement des comportements intelligents. Cette hypothèse était à la base de l'essor des réseaux de neurones artificiels dans le cadre de l'intelligence artificielle étudiant \textit{les moyens susceptibles de doter les systèmes informatiques de capacités intellectuelles comparables à celles des êtres humains} \cite{IA}. C'est ainsi que les chercheurs ont proposé de modéliser les activités neuronales à large échelle en se basant sur des modèles simplifiés de neurones individuels en interaction puisque l'analyse de la connectivité et la dynamique fonctionnelle dans les structures neuronales montre souvent des liens directs entre l'organisation neuronale et les réponses comportementales. Le but est d'étudier des processus de compétition et de coopération résultant des règles locales pour permettre d'expliquer des macro-dynamiques corticales et sous-corticales. Ce qui nous amène à la notion de \textit{l'\'emergence} qui affirme que l'on peut produire des comportements complexes (à un niveau macroscopique) et obtenir des propriétés imprévisibles par l'interaction entre des processus élémentaires simples (à un niveau microscopique). Nous avons choisi d'illustrer ce problème par un cas simple d'une tâche sensori-motrice, à savoir, l'initiation d'une saccade oculaire guidée par une cible visuelle. \\

Nous avons choisi le système visuo-moteur intensivement étudié par la communauté scientifique car il est à la fois complexe et particulier. Il est particulier car, d'une part, sa partie motrice est simple (6 muscles), d'autre part, il fait intervenir des structures (le colliculus supérieur) et des boucles particulières (circuit cortico-basal oculomoteur) qui le distinguent des autres systèmes sensori-moteurs. Il est complexe dans la mesure o\`u il fait intervenir cinq fonctions cognitives importantes: \\
\begin{itemize}  
\item La perception centrée sur l'objet (reconnaître et identifier des objets en traitant leurs propriétés visuelles intrinsèques).  
\item L'attention spatiale  (localiser rapidement les informations visuelles les plus pertinentes dans l'environnement). 
\item La prise de décision (sélectionner des cibles selon le contexte et la motivation).
\item La réalisation motrice (bouger les yeux pour exécuter les ordres moteurs).
\item L'organisation temporelle des comportements (coordonner les séquences de mouvements).\\
\end{itemize}

L'exemple que nous avons choisi illustre la fonction de l'attention spatiale. Nous avons abordé l'initiation des mouvements oculaires et la sélection intrinsèque des cibles visuelles en modélisant le flux visuel de la rétine vers une structure sous-corticale appelée le colliculus supérieur sans intégrer la réalisation motrice. Le modèle montre comment à partir d'une population homogène topologique avec un traitement séquentiel peuvent émerger des propriétés computationnelles intéressantes, en particulier un effet de ``winner-takes-all'' (le plus fort gagne).\\

\item[$\bullet$]Enfin, si on revient à la théorie de l'énaction, le traitement de l'information sensori-motrice est loin d'être un traitement purement séquentiel et les représentations mentales évoluent dans le temps. En effet, plusieurs boucles externes (résultant de l'interaction avec l'environnement) et internes (liées à la motivation, la mémoire, ...) interagissent par des flux asynchrones et hétérogènes. La question sur la manière de passer d'une représentation spatiale (perception passive) à une organisation temporelle des comportements qui prend en compte la variation du contexte interne et externe (perception active) reste ouverte. Plus particulièrement, comment une prise de décision est-elle assurée, à large échelle, lorsque des boucles différentes et des flux hétérogènes viennent biaiser le traitement séquentiel (i.e., entrée-représentation interne-sortie)? Si on retourne vers l'hypothèse de la reconnaissance visuelle active des objets, une droite est reconnue non pas en la comparant à une image de droite enregistrée dans la mémoire, mais plutôt par la séquence de saccades (mouvements) effectuées le long de la droite permettant de repérer par exemple l'invariance par translation. Il est donc intéressant de comprendre comment le choix des points à parcourir est réalisé, en particulier comment il est possible de diriger l'attention vers des points non-saillants pour permettre l'exploration de l'image.
Nous avons essayé de répondre à ces questions en étudiant et en modélisant une autre structure sous-corticale, les ganglions de la base. D'une part, cette structure  est impliquée dans une boucle oculomotrice, d'o\`u la pertinence d'un rapprochement avec le modèle précédent. D'autre part, cette structure qui implique plusieurs flux d'informations et différentes boucles d'interaction semble avoir un rôle actif dans la sélection de l'action. Nous proposons donc un modèle de sélection active et motivée qui montre comment le traitement séquentiel (par défaut) peut être biaisé et comment de nouvelles représentations peuvent êtres créées.\\

\end{itemize}

Des compétences en neurosciences et des données expérimentales ont été obtenues dans le cadre de la collaboration que nous entretenons avec nos collègues biologistes de l'INCM à Marseille. En effet, une partie de ces travaux de thèse s'intègre dans le cadre d'un projet ANR MAPS (Mapping, Adaptation, Plasticity and Spatial Computation) en collaboration entre l'INRIA Nancy Grand-Est/Loria, l'UMR Mouvement et Perception Marseille, l'INCM-CNRS Marseille et le LIRIS Lyon. Dans ce projet, par le biais de l'étude des propriétés de certaines structures spécifiques du cerveau envisagée sous plusieurs angles (les neurosciences, la modélisation, la psychologie expérimentale), nous avons cherché à mieux comprendre et modéliser les processus de traitement de l'information mis en oeuvre dans le cerveau, notamment au niveau des mécanismes temporels et spatiaux.\\


Nous commencerons, dans un premier chapitre, par étudier les problèmes liés aux formalismes de calcul utilisés. Nous allons définir, dans un premier temps, le cadre de modélisation en passant du neurone biologique au neurone artificiel. Ensuite, nous allons introduire l'approche connexionniste utilisée et aborder trois principaux aspects du cadre de modélisation, à savoir, l'intégration numérique, le calcul adaptatif et le calcul distribué. Nos contributions portent sur les problèmes de discrétisation dans le cadre de l'intégration numérique et sur les questions liées à l'évaluation synchrone dans le cadre du calcul distribué. Nous donnerons en particulier des résultats et des outils intéressant à appliquer dans le domaine des champs de neurones dynamiques.\\

Dans le second chapitre, nous donnerons une description du cadre biologique de nos travaux, à savoir le système oculomoteur. Nous examinerons les différents types de mouvements des yeux pour bien situer les saccades, qui sont des mouvements rapides et précis du globe oculaire permettant de focaliser la vision sur un point d'intérêt dans la scène visuelle et qui seront l'exemple de l'action (réponse motrice) traitée dans le chapitre suivant. Enfin, nous donnerons une vue globale du cerveau en décrivant les différentes structures corticales et sous-corticales impliquées dans le contrôle oculomoteur pour montrer la complexité des flux et des boucles qui peuvent être impliquées dans la gestion des saccades.\\

Dans le troisième chapitre, nous décrirons le modèle du colliculus supérieur que nous proposons comme exemple d'intégration sensori-motrice séquentielle. Nous montrerons qu'un traitement séquentiel basé sur des règles locales simples permet d'expliquer plusieurs propriétés et différents comportements sans avoir besoin d'imposer des entrées exogènes ou des feed-back. Deux hypothèses clés ont permis d'obtenir ces caractéristiques émergentes : la topographie colliculaire et le profil d'interactions latérales au sein de la population homogène. Il en résulte un système de sélection passive basé sur l'architecture intrinsèque. Un rapprochement intéressant avec les résultats biologiques a été mis en avant.\\

Dans le dernier chapitre, nous examinerons le cas de l'action basée sur l'interaction entre plusieurs boucles et flux d'informations à large échelle (entre plusieurs structures). Nous avons examiné le mécanisme de prise de décision en nous intéressant à la sélection motivée de l'action telle qu'elle semble se dérouler dans des structures neuronales sous-corticales appelées les ganglions de la base, en modélisant le circuit moteur les reliant au cortex. Nous avons proposé en particulier un mécanisme de création dynamique d'une représentation du contexte. Le rapprochement avec le circuit oculomoteur n'a pas encore été fait puisque nous avons favorisé l'étude de la fonction générale dans le cadre du circuit moteur qui est plutôt un circuit baso-cortical alors que le circuit oculomoteur est un cas particulier qui fait intervenir le colliculus supérieur. \\

Les chapitres seront suivis de discussions. Enfin, une synthèse des contributions et des perspectives est donnée dans une conclusion générale. \\

 
