\DontNumberThisInToc
\DontFrameThisInToc

\ChapterNoNumberCitation{Discussion}{Yes I can}{10cm}

\section{Contributions}
\subsection{No clock to rule them all}
\subsection{La sélection émerge de la topographie : un modèle du colliculus supérieur}
Nous avons introduit un modèle minimaliste de colliculus supérieur sur la base de d'un large ensemble de données biologiques. Ce modèle a été conçu  en s'appuyant sur un minimum d'hypothèses dans un cadre de calcul numérique distribué. Le comportement des saccades, à travers les différentes expériences, est une propriété émergente d'un calcul local et homogène. Si on examine de peu plus près le processus de sélection qui est effectuée lorsque le modèle reçoit en entrée deux stimuli identiques (mais
à deux endroits différents), on peut expliquer le choix des stimuli les plus proches à la région fovéale par la magnification corticale. Autrement dit, la cette magnification influence  profondément la topologie du réseau et par conséquent la saillance de tout stimuli présenté. Ce comportement de sélection
est donc très liée à la disposition spatiale et physique des unités de calcul. \\

\subsection{La sélection motivée : un modèle de ganglions de la base}
\section{Perspectives
\subsection{Cablage des modèles}
\subsection{Le modèle d'apprentissage}
\subsection{La spirale d'Haber}

\section{Conclusion}
