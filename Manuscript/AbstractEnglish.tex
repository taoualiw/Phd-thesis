\small {Within the context of enaction and a global approach to perception, we focused on the characteristics of neural computation necessary to understand the relationship between structures in the brain and their functions. We first considered computational problems related to the discretization of differential equations that govern the studied systems and the synchronous and asynchronous evaluation schemes. Then, we investigated a basic functional level: the transformation of spatial sensory representations into temporal motor actions within the visual-motor system. We focused on the visual flow from the retina to the superior colliculus to propose a minimalist model of automatic encoding of saccades to visual targets. This model, based on simple local rules (CNFT and logarithmic projection) in a homogeneous population and using a sequential processing, reproduces and explains several results of biological experiments. It is then considered as a robust and efficient basic model. Finally, we investigated a more general functional level by proposing a computational model of the basal ganglia motor loop. This model integrates sensory, motor and motivational flows to perform a global decision based on local assessments. We implemented an adaptive process for action selection and context encoding through an innovative mechanism that allows to form the basic circuit for other cortico-basal loops. This mechanism allows to create internal representations according to the enactive approach that opposes the computer metaphor of the brain. Both models have interesting dynamics to study from whether a biological point of view or a computational numerical one.}

\KeyWords {oculomotricity, neural networks, saccade, superior colliculus, basal ganglia, action selection, asynchronous computation, CNFT}
