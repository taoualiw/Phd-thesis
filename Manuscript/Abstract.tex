
\small{Dans le contexte de l'énaction et dans une approche globale de la perception, nous nous sommes intéressés à étudier calcul neuronal permettant de comprendre les relations entre les structures dans le cerveau et leurs fonctions. Nous avons d'abord examiné les problèmes calculatoires liés à la discrétisation des équations différentielles qui régissent les systèmes étudiés et aux schémas d'évaluation synchrones et asynchrones. Nous nous sommes, ensuite, intéressés à un niveau fonctionnel élémentaire: la transformation de représentations sensorielles spatiales en actes moteurs temporels dans le cadre du système visuo-moteur. Nous avons proposé un modèle minimaliste d'encodage automatique des cibles visuelles de saccades qui se concentre sur le le flux visuel de la rétine vers le colliculus supérieur. Ce modèle, basé sur sur des règles locales simples au sein d'une population homogène, permet de reproduire et d'expliquer plusieurs résultats d'expériences biologiques ce qui en fait un modèle de base efficace et robuste. Enfin, nous avons abordé un niveau fonctionnel plus global en proposant un modèle de la boucle motrice des ganglions de la base permettant d'intégrer des flux sensoriels, moteurs et motivationnels en vue d'une décision globale reposant sur des évaluations locales. Ce modèle met en exergue un processus adaptatif de sélection de l'action et d'encodage de contexte via des mécanismes originaux lui permettant en particulier de constituer la brique de base pour les autres boucles cortico-basales. Les deux modèles présentent des dynamiques intéressantes à étudier que ce soit d'un point de vue biologique ou d'un point de vue informatique computationnel.}

\KeyWords{oculomotricité, réseaux de neurones, saccade, colliculus supérieur, ganglions de la base, sélection de l'action, calcul asynchrone, CNFT}
